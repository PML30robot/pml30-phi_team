
\subsubsection{04.11.14}

\begin{enumerate}
	\item Время начала и окончания собрания:
	14:00 – 20:30
	\item Цели собрания:
	\begin{enumerate}
	  \item	Переделать конструкцию лебедки.
	  
	  \item	Подсоединить энкодер к одному из приводов лебедки.
	  
	  \item	Добавить в программу управления лебедкой ограничения ее движения.
	  
    \end{enumerate}
    
	\item Проделанная работа:
	\begin{enumerate}
	  \item	Механизм лебедки был изменен в соответствии с идеями, выдвинутыми на предыдущем занятии.
      
      \item	Энкодер был установлен на левый привод лебедки.
      
      \begin{figure}[H]
      	\begin{minipage}[h]{0.47\linewidth}
      		\center{\includegraphics[scale=0.3]{days/images/missing_image}}
      		\caption{Окончательная версия механизма лебедки}
      	\end{minipage}
      	\hfill
      	\begin{minipage}[h]{0.47\linewidth}
      		\center{\includegraphics[scale=0.3]{days/images/missing_image}}
      		\caption{Энкодер}
      	\end{minipage}
      \end{figure}
      
      \item	В программу были добавлены ограничения движения лебедки. Таким образом, если показания энкодера превышали допустимое значение, лебедка автоматически останавливалась.
      
      \item	Испытания лебедки прошли успешно. Новая конструкция лебедки была надежна и не имела никаких проблем с раздвиганием подъемника.
      
    \end{enumerate}
    
	\item Итоги собрания: 
	\begin{enumerate}
	  \item	Работа над механизмом лебедки завершена.
	  
	  \item	Испытания лебедки прошли успешно.
	  
    \end{enumerate}
    
	\item Задачи для последующих собраний:
	\begin{enumerate}
	  \item	Продолжить работать над механизмами ковша и захвата корзин.
	  
    \end{enumerate}     
\end{enumerate}
\fillpage

