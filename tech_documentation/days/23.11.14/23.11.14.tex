\subsubsection{23.11.14 (Соревнования)}
\begin{center}
	3-ий день соревнований "Робофест-Юг"
\end{center}
Сегодня проходили финальные матчи.

Результаты соревнований:
\begin{enumerate}
	\item По результатам квалификационных матчей в первую четверку мы не попали.
	
	\item В финальные игры мы не вышли, поскольку мало общались с другими командами, и в результате не были выбраны никем из первой четверки.
	
	\item Мы заняли 1 место в номинации "Защита инженерной книги". В качестве приза за первое место нам выдали по одному основному и дополнительному набору деталей TETRIX.
	
	\item Благодаря тому, что мы победили в номинации "Защита инженерной книги", наша команда получила квоту на участие во всероссийских соревнованиях по FTC, которые пройдут ближе к концу соревновательного сезона в Москве.
	
\end{enumerate}

Внесенные доработки:
\begin{enumerate}
	\item Была начата программа для автономного периода, включающая в себя перемещение подвижной корзины в зону парковки. Конечная стратегия автономного периода выглядит так: робот съезжает с пандуса, подъезжает к корзине, захватывает ее, закидывает в нее автономные мячи, поворачивается в сторону зоны парковки и едет в зону парковки вместе с корзиной. Однако, при испытании было обнаружено, что робот может повернуться на угол больший некоего порогового, но не может повернуться на меньший. После возвращения домой нам предстоит разобраться с этой проблемой.
\end{enumerate}

Подведение итогов:
\begin{enumerate}
  \item Успешность выступления на соревнованиях:
  \begin{enumerate}
	\item По результатам игры мы не прошли в финал и не заняли призовых мест.
	
	\item Мы заняли первое место в категории "Защита инженерной книги".
	
	\item Мы получили квоту на участие во всероссийских соревнованиях.
	
  \end{enumerate}
  
  \item Полезные технические решения, которые мы почерпнули у других команд:
  \begin{enumerate}
  	
	\item В качестве реек на подъемнике можно использовать конструкционные профили, в пазы которых вставляются детали, повторяющие их внутреннее сечение так, что профиль способен свободно скользить относительно детали. Внутренняя деталь может быть изготовлена на 3D принтере или вручную. Также могут использоваться оригинальные квадратные гайки, которые обычно служат для скрепления конструкционных профилей вместе.
	
	\item Для того, чтобы робот не подпрыгивал при развороте, можно реализовать ходовую часть из 3 пар колес, где средняя пара расположена чуть ниже остальных. Благодаря такому расположению колес робот в любой момент времени опирается только на две соседних пары и направление вращения колес почти совпадает с касательными к окружности, вокруг которой робот вращается.
	
	\item Если необходимо развить очень большое усилие по прямой, наиболее компактным и надежным решением является использование болтовой червячной передачи - передачи, построенной на взаимодействии болта и гайки. В то время, как гайка остается неподвижной, болт проворачивается мотором и двигается относительно гайки вперед или назад. Такая конструкция удобна, например, при создании ножничного подъемника, а также способна заменить собою гидравлический привод при решении других инженерных задач.
	
	\item Если существует необходимость в точном измерении перемещения робота, а колеса проскальзывают, можно создать независимую платформу (на 3-х омни колесах, обеспечивающих постоянное сцепление с полем отсутствие проскальзывания), оснащенную энкодерами, которая прикреплена к роботу эластичным соединением и двигается совместно с ним (во внутренней области робота, а не рядом с ним, так как иначе она может быть повреждена). Ориентируясь по показаниям энкодеров, установленных на платформе, можно точнее направлять робота. Минусы - сложность и громоздкость конструкции.
	
	\item Одна из команд с соревнований по роботизированному баскетболу имела специальное приспособление - колесо, раскручивающееся до около 1000 оборотов в минуту - для ускорения баскетбольных мячей. Если мы установим такое колесо на механизм захвата мячей, то оно сможет подбрасывать их вверх на большую высоту (возможно даже 120 см), что позволит нам закидывать мячи в корзины без помощи подъемника. Поскольку эта идея труднореализуема, мы займемся ею только при наличии свободного времени и отдельным модулем, не разбирая имеющейся конструкции.
	
	\item Одна из команд имела приспособление, напоминающее откидывающуюся баскетбольную корзину, позволяющее выкидывать мячи через круглое отверстие в днище ковша. В этом случае мячи выпадали из ковша вертикально и точность их забрасывания в подвижные корзины была очень велика.
	
  \end{enumerate}
  
  \item Наши ошибки и недостатки конструкции:
  \begin{enumerate}
  	\item Мы уделяли недостаточно времени тренировкам. Нужно больше тренироваться.
  	
  	\item Мы не смогли эффективно реализовать себя в общении с другими командами во время договоров о вступлении в альянс. Нам необходимо больше общаться с другими командами.
  	
  	\item В ходе тренировок на поле мы выяснили, что сбить упор очень трудно и устройство, которое мы предполагали сделать (вертушка из сервопривода свободного вращения), с этим не справится.
  	
  	\item Также мы решили отказаться от установки на робота омни-колес, поскольку роботы на омни-колесах испытывают большие трудности с заездом на пандус, что значительно ограничивает сферу их применения.
  	
  \end{enumerate}
  
  \item Задачи для последующих собраний:
  \begin{enumerate}
  	\item Усовершенствовать программы автономного периода при старте и с пандуса, и из зоны парковки.
  	
  	\item Переместить ось, вокруг которой поворачивается ковш при опрокидывании, на верхнюю часть последней мебельной рейки для того, чтобы ковш было возможно опрокидывать при любом положении подъемника, а не только при полностью раздвинутом подъемнике.
  	
  	\item Поставить на ковш механизм, который будет направлять шары, падающие в корзину, вертикально. Это повысит качество забрасывания мячей в подвижные корзины т.к. мячи будут лететь ровнее и не будут отскакивать от корзины, ударяясь о ее края.
  	
  	\item Установить на механизм лебедки 4 привода вместо двух для увеличения скорости раздвигания подъемника.
  	
  	\item Доработать захват шаров: заменить стяжки на более надежно захватывающее и менее хрупкое (во время соревнований около трети стяжек сломались) устройство. Например, на полукруглые обрезки от пластиковых бутылок.
  	
  	\item Отработать навыки эффективного управления роботом у операторов.
  	
  	\item Распутать провода питания приводов и модулей и провести провода аккуратно.
  	
  \end{enumerate}
  
\end{enumerate}
\fillpage