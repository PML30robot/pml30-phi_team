\subsubsection{23.11.14 (Соревнования)}
\begin{center}
	3-ий день соревнований "Робофест-Юг"
\end{center}
Сегодня проходили финальные матчи.

Результаты соревнований:
\begin{enumerate}
	\item По результатам квалификационных матчей в первую четверку мы не попали.
	
	\item В финальные игры мы не вышли, поскольку мало общались с другими командами, и в результате не были выбраны никем из первой четверки.
	
	\item Мы заняли 1 место в номинации "Защита инженерной книги".
\end{enumerate}

Внесенные доработки:
\begin{enumerate}
	\item Была начата программа для автономного периода, включающая в себя перемещение подвижной корзины в зону парковки. Конечная стратегия автономного периода выглядит так: робот съезжает с пандуса, подъезжает к корзине, захватывает ее, зкидывает в нее автономные мячи, поворачивается в сторону зоны парковки и едет в зону парковки вместе с корзиной. Однако, при испытании было обнаружено, что робот может повернуться на угол больший некоего порогового, но не может повернуться на меньший. После возвращения домой нам предстоит разобраться с этой проблемой.
\end{enumerate}

Подведение итогов:
\begin{enumerate}
  \item Успешность выступления на соревнованиях:
  \begin{enumerate}
	\item По результатам игры мы не прошли в финал и не заняли призовых мест.
	
	\item Мы заняли первое место в категории "Защита инженерной книги".
	
  \end{enumerate}
  
  \item Полезные технические решения, которые мы подчерпнули у других команд:
  \begin{enumerate}
	\item Другие рейки.
	
	\item 6 колес.
	
	\item Винтовая передача.
	
	\item Свободный энкодер.
	
	\item Ускоритель мячей.
	
	\item Точность забрасывания.
	
  \end{enumerate}
  
  \item Наши ошибки и недостатки конструкции:
  \begin{enumerate}
  	\item Мы уделяли недостаточно времени тренировкам. Нужно больше тренироваться.
  	
  	\item Мы не смогли эффективно реализовать себя в общении с другими командами во время договоров о вступлении в альянс. Нам необходимо больше общаться с другими командами.
  	
  \end{enumerate}
  
  \item Задачи для последующих собраний:
  \begin{enumerate}
  	\item Усовершенствовать программы автономного периода при старте и с пандуса, и из зоны парковки.
  	
  	\item Переместить ось, вокруг которой поворачивается ковш при опрокидывании, на верхнюю часть последней мебельной рейки для того, чтобы ковш было возможно опрокидывать при любом положении подъемника, а не только при полностью раздвинутом подъемнике.
  	
  	\item Поставить на ковш механизм, который будет направлять шары, падающие в корзину, вертикально. Это повысит качество забрасывания мячей в подвижные корзины т.к. мячи будут лететь ровнее и не будут отскакивать от корзины, ударяясь о ее края.
  	
  	\item Установить на механизм лебедки 4 привода вместо двух для увеличения скорости раздвигания подъемника.
  	
  	\item Доработать захват шаров: заменить стяжки на более надежно захватывающее и менее хрупкое (во время соревнований около трети стяжек сломались) устройство. Например, на полукруглые обрезки от пластиковых бутылок.
  	
  	\item Отработать навыки эффективного управления роботом у операторов.
  	
  \end{enumerate}
  
\end{enumerate}
\fillpage