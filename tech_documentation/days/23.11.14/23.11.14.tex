\subsubsection{23.11.14 (Соревнования)}
\begin{center}
	3-ий день соревнований "Робофест-Юг"
\end{center}
Сегодня проходили финальные матчи.
Внесенные доработки:
\begin{enumerate}
	\item Была начата программа для автономного периода, включающая в себя перемещение подвижной корзины в зону парковки. Конечная стратегия автономного периода выглядит так: робот съезжает с пандуса, подъезжает к корзине, захватывает ее, кладет туда автономные мячи, поворачивается на небольшой угол и едет в зону парковки вместе с корзиной. Однако, при испытании было обнаружено, что робот может повернуться на угол больший некоего порогового, но не может повернуться на меньший
\end{enumerate}
Результаты:
\begin{enumerate}
	\item По результатам квалификационных матчей в первую четверку не попали
	\item В финальные игры не вышли. Причины: мало общались с другими командами, из-за чего мы не были выбраны никем из первой четверки
	\item 1 место в номинации "Инженерная книга"
\end{enumerate}
Выводы:
\begin{enumerate}
	\item Больше времени уделять тренировкам
	\item Больше общаться с другими командами
\end{enumerate}
Задачи для последующих собраний:
\begin{enumerate}
	\item Доделать автономный период
	\item Поставить на ковш механизм, который будет направлять шары, падающие в корзину из ковша, перпендикулярно поверхности земли. Это повысит точность забрасывания мячей в подвижные корзины т.к. шары не будут лететь мимо корзины из-за ударов о ее края
	\item Установить на раздвигание подъемника 4 мотора вместо двух для увеличения скорости подъема шаров
	\item Доработать захват шаров: заменить стяжки на что-нибудь более надежно захватывающее и менее хрупкое (во время соревнований около трети стяжек сломались). Например, на полукруглые обрезки от пластиковых бутылок
	\item Тренироваться
\end{enumerate}
\fillpage