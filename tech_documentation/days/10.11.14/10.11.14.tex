
\subsubsection{10.11.14}

\begin{enumerate} 
	\item Время начала и окончания собрания:\newline
	17:30 - 20:30
	\item Цели собрания:\newline
	\begin{enumerate}
		\item Выбрать оптимальный вариант конструкции прицепа.\newline
		
		\item Начать создание механизма прицепа.\newline
		
	\end{enumerate}
	
	\item Проделанная работа:\newline
	\begin{enumerate}
		\item Предпочтение было отдано конструкции с вертикальными рейками, как наиболее простой и компактной.\newline
		
		\item Было решено, что для приведения механизма в действие будут использоваться два сервопривода: один будет опускать рейки на подставку корзины, а второй - раздвигать обе рейки. Сборка механизма прицепа начата, но не завершена.\newline
		
		\item Сегодня была начата работа над декоративной стороной нашего проекта - мы начали вышивать эмблему нашей школы -- ФМЛ№30.\newline
		
	    \begin{figure}[H]
			\begin{minipage}[h]{0.2\linewidth}
				\center  
			\end{minipage}
			\begin{minipage}[h]{0.6\linewidth}
				\center{\includegraphics[width=1\linewidth,height=0.9\measurepage]{days/10.11.14/images/01}}
				\caption{Эмблема нашей команды}
			\end{minipage}
		\end{figure}
		
	\end{enumerate}
	
	\item Итоги собрания: \newline
	\begin{enumerate}
		\item Конструкция прицепа выбрана.\newline
		
		\item Механизм прицепа частично собран.\newline
		
		\item Частично создана эмблема нашей команды, которая будет установлена на роботе.\newline
		
	\end{enumerate}
	
	\item Задачи для последующих собраний:\newline
	\begin{enumerate}
		\item Завершить сборку конструкции прицепа.\newline
		
		\item Написать программу для управления прицепом.\newline
		
		\item Закончить эмблему и прикрепить ее к роботу.\newline
		
	\end{enumerate}     
\end{enumerate}

\fillpage

