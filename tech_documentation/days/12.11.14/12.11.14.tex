
\subsubsection{11.11.14} 

\begin{enumerate} 
	\item Время начала и окончания собрания:\newline
	17:00 - 20:30
	\item Цели собрания:\newline
	\begin{enumerate}
		\item Завершить работу над механизмом прицепа.\newline
		
		\item Добавить в программу управления роботом управление системой захвата подвижных корзин.\newline
		
		\item Написать программу раздельного управления роботом с двух джойстиков.\newline
		
	\end{enumerate}
	
	\item Проделанная работа:\newline
	\begin{enumerate}
		\item Механизм прицепа был завершен.\newline
		
		\begin{figure}[H]
			\begin{minipage}[h]{0.2\linewidth}
				\center  
			\end{minipage}
			\begin{minipage}[h]{0.6\linewidth}
				\center{\includegraphics[width=1\linewidth,height=0.9\measurepage]{days/11.11.14/images/01}}
				\caption{Готовый механизм прицепа}
			\end{minipage}
		\end{figure}
		
		\item Программа управления механизмом прицепа не реализована.\newline
		
		\item Сегодня было выбрано окончательное место для NXT-блока. Пока он был временно закреплен7 на скотч, но в будущем мы планируем закрепить его надежнее.\newline
		
		\begin{figure}[H]
			\begin{minipage}[h]{0.2\linewidth}
				\center  
			\end{minipage}
			\begin{minipage}[h]{0.6\linewidth}
				\center{\includegraphics[width=1\linewidth]{days/11.11.14/images/01}}
				\caption{Место крепления NXT-блока}
			\end{minipage}
		\end{figure}
		
		\item Мы обратили внимание на то, что провод сервопривода, закрепленного на подъемнике, в то время, когда подъемник сложен, касается пола и может помешать движению ковша. Было решено создать специальную катушку, работающую по принципу рулетки и сматывающую провод тогда, когда он не находится в натяжении, либо просто закрепить провод в нескольких местах подъемника таким образом, чтобы он не создавал помех в работе ковша и подъемника.\newline
		
		\item В связи с тем, что нам иногда может потребоваться испытать один из узлов или провернуть с помощью программы приводы, которые трудно провернуть руками, например, для ремонта или замены деталей, было решено создать специальную вспомогательную программу, которая позволила бы управлять отдельными приводами и сервоприводами при помощи кнопок, встроенных в NXT-блок, не прибегая к управлению роботом с джойстика. Такая программа была бы очень удобна в таких случаях, как, например, при внезапной потере связи с роботом по Bluetooth или Samantha, когда необходимо привести механизм подъемника в начальное положение.\newline
		
		\item Программа раздельного управления роботом создана, но не испытана. В новой программе первый оператор отвечает за все, кроме движения, а второй - соответственно за движение робота.\newline
		
		\item На общекомандном обсуждении была разработана идея механизма, служащего для сбивания упора у центральной стойки в автономном режиме: на робота должен быть установлен сервопривод свободного вращения, на котором будут закреплены две цепочки из балок из конструктора LEGO-NXT, скрепленных между собой последовательно таким образом, что каждые две из них соединены между собой только одним штифтом. В сложенном состоянии такая цепочка не будет занимать много места, но с началом вращения, благодаря центробежной силе, распрямится и будет напоминать плеть. После того, как цепь распрямится, для сбития упора центральной стойки роботу будет достаточно проехать от него на расстоянии действия механизма. Это намного проще и удобнее, чем писать программу для поиска упора по ИК-датчику.\newline
		
		\begin{figure}[H]
			\begin{minipage}[h]{0.2\linewidth}
				\center  
			\end{minipage}
			\begin{minipage}[h]{0.6\linewidth}
				\center{\includegraphics[width=1\linewidth]{days/11.11.14/images/01}}
				\caption{Идея для механизма сбивания упора}
			\end{minipage}
		\end{figure}
		
	\end{enumerate}
	
	\item Итоги собрания: \newline
	\begin{enumerate}
		\item Механизм прицепа завершен.\newline
		
		\item Программа управления прицепом не реализована.\newline
		
		\item Программа раздельного управления роботом создана.\newline
		
		\item NXT-блок закреплен на роботе.\newline
		
		\item Была разработана концепция механизма сбивания упора.\newline
		
	\end{enumerate}
	
	\item Задачи для последующих собраний:\newline
	\begin{enumerate}
		\item Испытать программу раздельного управления роботом с двух джойстиков.\newline
		
		\item Включить в программу управления роботом программу управления механизмом прицепа.\newline
		
		\item Закрепить провод сервопривода, расположенного на подъемнике так, чтобы он не мешал работе подъемника и ковша.\newline
		
		\item Написать вспомогательную программу для управления узлами робота без использования компъютера и джойстика.\newline
		
		\item Собрать механизм сбивания упора и испытать его в действии.\newline
		
	\end{enumerate}     
\end{enumerate}

\fillpage

