
\subsubsection{16.11.14}

\begin{enumerate} 
	\item Время начала и окончания собрания:
	19:00 - 20:30
	\item Цели собрания:
	\begin{enumerate}
		\item Завершить работу над конструкцией ковша.
		
		\item Закрепить ковш на механизме опрокидывания ковша.
		
	\end{enumerate}
	
	\item Проделанная работа:
	\begin{enumerate}
		\item Заготовка для ковша была изменена: верхняя ее часть была свернута таким образом, что образовывала трубу, по которой будут скатываться мячи во время опрокидывания ковша назад. Нижняя осталась без изменений.
		
		\item Было решено закрепить внутри металлического каркаса трубы пластмассовую бутылку для того, чтобы мячам было легче скатываться по трубе. Сегодня мы не могли этого сделать, поскольку у нас не было пластмассовой бутылки, поэтому было решено реализовать эту идею на следующем занятии.
		
		\item Ковш был закреплен на механизме опрокидывания ковша.
		
		\item После закрепления ковша на роботе выяснилось, что из-за поперечной балки, расположенной в передней части робота, он не опускается вниз до конца. Тогда балка была заменена на более тонкую, которая не препятствовала движению ковша.
		
	    \begin{figure}[H]
			\begin{minipage}[h]{0.47\linewidth}
				\center{\includegraphics[width=1\linewidth,height=0.9\measurepage]{days/10.11.14/images/01}}
				\caption{Ковш в начальном положении}
			\end{minipage}
			\hfill
			\begin{minipage}[h]{0.47\linewidth}
				\center{\includegraphics[width=1\linewidth,height=0.9\measurepage]{days/10.11.14/images/01}}
				\caption{Ковш в перевернутом положении}
			\end{minipage}
		\end{figure}
		
	\end{enumerate}
	
	\item Итоги собрания:
	\begin{enumerate}
		\item Каркас ковша сконструирован и установлен на робота.
		
	\end{enumerate}
	
	\item Задачи для последующих собраний:
	\begin{enumerate}
		\item Доработать конструкцию ковша.
		
		\item Протестировать работу ковша.
		
	\end{enumerate}     
\end{enumerate}

\fillpage

