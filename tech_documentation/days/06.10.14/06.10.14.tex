\textsl{}
\subsubsection{06.10.14}

\begin{enumerate}
	\item Время начала и окончания собрания:
	21:00 - 21:30
	\item Цели собрания:
	\begin{enumerate}
	  \item Начать писать программу для управления роботом с джойстика.
	  
    \end{enumerate}
	\item Проделанная работа:
	\begin{enumerate}
	  \item Для проверки способностей ходовой было написано две программы движения робота – по прямой и вокруг своей оси. При движении по прямой робот показал блестящие результаты, поскольку почти не отклонился от изначальной траектории за все время движения. При вращении вокруг своей оси робот сильно дребезжал, поскольку из-за высокого коэффициента трения его колеса не могли проскальзывать по полу и подпрыгивали, но в целом это никак не влияло на точность поворота. Робот вращался точно вокруг своего центра тяжести, однако последний находился не в центре робота, а ближе к задней части. Возможно, для удобства управления следует переместить центр тяжести ближе к линии пересечения диагоналей квадрата, в углах которого расположены колеса. Тогда место, необходимое на разворот, будет меньше и мы будем меньше мешать союзникам.
      
      \item  В результате обсуждения типа подъемника был выбран вариант с раздвижными мебельными рейками, основание которых жестко зафиксировано на каркасе робота. Эта конструкция наиболее надежная из всех описанных выше, а также самая простая в исполнении.
      
    \end{enumerate}
    
	\item Итоги собрания: 
	\begin{enumerate}
	  \item  Программа управления с джойстика пока не реализована.
	  
      \item  Выбран тип подъемника для мячей.
    \end{enumerate}
    
	\item Задачи для последующих собраний:
	\begin{enumerate}
	  \item Купить мебельные рейки для создания подъемника.

    \end{enumerate}     
\end{enumerate}
\fillpage
