
\subsubsection{28.10.14}

\begin{enumerate}
	\item Время начала и окончания собрания:\newline
	17:00 - 19:00
	\item Цели собрания:\newline
	\begin{enumerate}
	  \item Закончить работу над механизмом прицепа.\newline
	  
    \end{enumerate}
    
	\item Проделанная работа:\newline
	\begin{enumerate}
	  \item Сервопривод, который вращает балку, должен быть закреплен как можно ниже для максимальной точности захвата подвижной корзины.\newline
      
      \item Решено было закрепить сервопривод следующим образом: просверлить в балке в задней части робота отверстие диаметром с вал сервопривода для того, чтобы разместить сервопривод так, чтобы он не выходил за пределы корпуса робота (иначе робот не укладывался бы в регламентированные размеры) и в то же время мог свободно вращаться.\newline
      
      \item Отверстие было просверлено.\newline
      
      \begin{figure}[H]
      	\begin{minipage}[h]{0.47\linewidth}
      		\center{\includegraphics[height=0.9\measurepage,width=1\linewidth]{days/28.10.14/images/01}}
      		\caption{Сервопривод}
      	\end{minipage}
      	\hfill
      	\begin{minipage}[h]{0.47\linewidth}
      		\center{\includegraphics[height=0.9\measurepage,width=1\linewidth]{days/28.10.14/images/01}}
      		\caption{Отверстие для сервопривода}
      	\end{minipage}
      \end{figure}
      
    \end{enumerate}
    
	\item Итоги собрания: \newline
	\begin{enumerate}
	  \item Придумана и частично реализована схема закрепления сервопривода для прицепа.\newline
	  
    \end{enumerate}
    
	\item Задачи для последующих собраний:\newline
	\begin{enumerate}
	  \item Закончить работу над механизмом прицепа.\newline
	  
    \end{enumerate}     
\end{enumerate}

\fillpage
