
\subsubsection{28.10.14}

\begin{enumerate}
	\item Время начала и окончания собрания:
	17:00 - 19:00
	\item Цели собрания:
	\begin{enumerate}
	  \item Закончить работу над механизмом захвата корзин.
	  
    \end{enumerate}
    
	\item Проделанная работа:
	\begin{enumerate}
	  \item Сервопривод, который вращает балку должен быть закреплен как можно ниже для максимальной точности захвата.
      
      \item Решено было закрепить сервопривод следующим образом:просверлить в балке в задней части робота отверстие диаметром с вал сервопривода для того, чтобы разместить сервопривод так, чтобы он не выходил за пределы корпуса робота (иначе робот не укладывался бы в регламентированные размеры) и в то же время мог свободно вращаться.
      
      \item Отверстие было просверлено.
      
      \begin{figure}[H]
      	\begin{minipage}[h]{0.47\linewidth}
      		\center{\includegraphics[scale=0.3]{days/images/missing_image}}
      		\caption{Сервопривод}
      	\end{minipage}
      	\hfill
      	\begin{minipage}[h]{0.47\linewidth}
      		\center{\includegraphics[scale=0.3]{days/images/missing_image}}
      		\caption{Отверстие для сервопривода}
      	\end{minipage}
      \end{figure}
      
    \end{enumerate}
    
	\item Итоги собрания: 
	\begin{enumerate}
	  \item Придуман и частично реализована схема закрепления сервопривода для механизма захвата корзин.
	  
    \end{enumerate}
    
	\item Задачи для последующих собраний:
	\begin{enumerate}
	  \item Закончить работу над механизмом захвата корзин.
	  
    \end{enumerate}     
\end{enumerate}

\fillpage
