\subsubsection{07.02.15}
\begin{enumerate}
	
	\item Время начала и окончания собрания: 16:30 - 20:00.
	
	\item Цели собрания: 
	\begin{enumerate}
		
		\item Тренироваться закидывать мячи в корзину 90 см. Постараться улучшить свое время.
		
	\end{enumerate}

	\item Проделанная работа:
	\begin{enumerate}
		
		\item Сегодня мы все занятие тренировались, но в среднем время наполнения корзины оставалось прежним - 2,5 минуты. В связи с этим, мы решили обсудить стратегию наших действий в управляемом периоде, и пришли к выводу, что наилучшей стратегией будет такая:
			\begin{enumerate} 
				\item Захватываем корзину 90 см (обычно, по итогам автономного периода, она нами уже будет захвачена).
				 
				\item Набираем по 5 больших мячей и 2 раза забрасываем по 5 мячей в корзину 90 см. 
				
				\item Завозим корзину 90 см в зону парковки (так безопаснее, поскольку завозя ее на пандус, мы можем ее уронить и потерять все очки). 
				
				\item Когда остается 45 секунд, мы начинаем набирать мячи для закидывания в центральную корзину (4-5 больших), затем подъезжаем к ней и ждем финала. 
				
				\item Как только начинается финал, мы опрокидываем ковш и забрасываем мячи в центральную корзину.
				
				\item Если остается более 15 секунд, мы едем на пандус, а если меньше - то в зону парковки, чтобы заработать хотя бы 10 очков.
				
			\end{enumerate}
		
	\end{enumerate}
	
	\item Итоги собрания:
	\begin{enumerate}
		
		\item Разработана стратегия управляемого периода.
		
	\end{enumerate}
	
	\item Задачи для последующих собраний:
	\begin{enumerate}
		
		\item Продолжить тренировки.
		
		\item Реализовать программу автономного периода с пандуса.
		
		\item Реализовать защиту колес от наезда на мячи и палку-упор.
			
	\end{enumerate}
\end{enumerate}
\fillpage
