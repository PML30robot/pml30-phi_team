\subsubsection{11.12.14}

\begin{enumerate}
	\item Время начала и окончания собрания: 17:30 - 23:30
	\item Цели собрания:
	\begin{enumerate}
		\item Установить стальные перекладины на подъемник
		
		\item Добавить второй сервопривод на опрокидывание ковша
		
		\item Упаковать робота для транспортировки на соревнования
	\end{enumerate}
	\item Проделанная работа:
	\begin{enumerate}
		\item Установлены две из трех стальных перекладин. В самом низу решено было оставить аллюминиевую т.к. в стальную невозможно было вставить трубку из-за ее чуть большего диаметра (примерно на 0.1мм), в то время как аллюминиевая ось с трубкой почти не изгибалась и к тому же из-за большего диаметра трубки и ее способности вращаться на оси уменьшалась сила трения
		
		\item Два сервопривода оказались не способны опрокинуть ковш и поэтому было решено переместить их на старое место
		
		\item Робот упакован для транспортировки
		
	\end{enumerate}
	\item Итоги собрания:
	\begin{enumerate}
		\item Стальные перекладины установлены
		
		\item Сервоприводы, опрокидывющие ковш было решено переместить на прежнее место
		
	\end{enumerate}
	\item Задачи для последующих собраний:
	\begin{enumerate}	
		\item Переместить на прежнее место сервоприводы, опрокидывающие ковш
		
		\item Потренироваться в управлении роботом
		
		\item Поставить на ковш механизм, который будет направлять шарики перпендикулярно земле
		
	\end{enumerate}
\end{enumerate}