\subsubsection{23.11.14 (Competition)}
\begin{center}
	 3-nd day of competition "Robofest-South"
\end{center}
Today there where final matches.

Results of competition:
\begin{enumerate}
	\item We didn't hit to "top 4" by the results of qualifying matches.
	
	\item We didn't take part in the final matches because we communicate with other team too little.
	
	\item We won in the nomination "The best engineering book". So we got two TETRIX MAX sets: main set and resourse one.
\end{enumerate}

Improvements that were done:
\begin{enumerate}
	\item It was started programme of autonomous period that include exit from the ramp throwing the autonomous balls to 60cm goal and moving it to the parking zone. But it was turned out that robot can turn to angle that bigger than a threshold but can't turn to smaller. We need to solve this problem after returning.
\end{enumerate}

Results:
\begin{enumerate}
  \item Successfully of performances on the competition:
  \begin{enumerate}
	\item We didn't take the prizes by the results of matces.
	
	\item We took the first place in the nomination "The best engineering book".
	
	\item We made it to regionals that will be in Moscow 12th-13th of February.
	
  \end{enumerate}
  
  \item Useful ideas that we took from another teams:
  \begin{enumerate}
  	
  	\item We met the captain of the team "Stuy Fission 310" (Stuversant High School, New York) James Chin. We had an interesting conversation and got some useful ideas:
  	\begin{enumerate}
  		\item We can use construction profiles instead of furniture slats in the construction of the lift. They are more reliable than slats.
  		
  		\item Wheel base with 3 pairs of standard wheels. The central pair is lower than others. So in each period of time robot stands on two pairs. It allows to it turn around without problems because the directions of their rotation are closer to the tangent to a circle around which robot turns. 
  		
  		\item When we need hydraulic drive we can to change it by screw with a nut. When nut rotates the screw moves on a straight direction.  
  	\end{enumerate}

	
	\item We looked robots for basketball competition "BasketBot" and noticed some ideas:
	\begin{enumerate}
		\item They had a mechanism for throwing balls. It consists of wheel that rotates with high speed. This wheel accelerates balls and they fly on a big heght. So we thought that we can use this mechanism for rising balls and throwing them to rolling goal. It will work faster than the construction with the bucket.  
	\end{enumerate}
	
	\item One of the team had a fixture that resemble a folding basketball basket that directs balls vertically. It allows much increase accuracy of throwing balls to the goals.
	
  \end{enumerate}
  
  \item Our mistakes and disadvantages of construction:
  \begin{enumerate}
  	\item We did not spend enough trainings. It need to train  more.
  	
  	\item We communicate with other team too little. It didn't allow to us to agree with ohter team on accession to the alliance.
  	
  	\item It was turned out that it hard to knock down the fence and mechanism that we planned to make doesn't cope with it.
  	
  	\item We decided to refuse from installation mechanum wheels because robot with mechanum wheels has a lot of problems with riding to the ramp.
  	
  \end{enumerate}
  
  \item Tasks for the next meetings:
  \begin{enumerate}
  	\item To improve the programmes of autonomous period.
  	
  	\item To move the MOB to the top of the slat in order to we can overturns the bucket when the lift is in any position.
  	
  	\item To install mechanism that will direct balls vertically. It will increase the accuracy of throwing balls to goals because balls will fall straightly.
  	
  	\item To install 4 motors instead 2 on the mechanism of extracting the lift (hereinafter it will call MEL).
  	
  	\item To improve the gripper for balls: change the couplers to something that captures the balls more securely and less fragile (a lot of couplers were broke during the competition). For example pieces of plastic bottle.
  	
  	\item To practice the skills of effective control of robot.
  	
  	\item To unravel the wires of power of motors and controllers and to hold their more gently.
  	
  \end{enumerate}
  
\end{enumerate}
\fillpage